%++++++++++++++++++++++++++++++++++++++++
% Don't modify this section unless you know what you're doing!
\documentclass[letterpaper,12pt]{article}
\usepackage{authblk}
\usepackage{tabularx} % extra features for tabular environment
\usepackage{amsmath, amssymb}  % improve math presentation
\usepackage{amsfonts}
\usepackage{graphicx} % takes care of graphic including machinery
\usepackage[margin=1in,letterpaper]{geometry} % decreases margins
\usepackage{cite} % takes care of citations
\usepackage[final]{hyperref} % adds hyper links inside the generated pdf file
\hypersetup{
    colorlinks=true,       % false: boxed links; true: colored links
    linkcolor=blue,        % color of internal links
    citecolor=blue,        % color of links to bibliography
    filecolor=magenta,     % color of file links
    urlcolor=blue         
}
\usepackage{float}
\usepackage{listings}
\usepackage[titletoc,title]{appendix}
\usepackage{siunitx}
\usepackage[table]{xcolor}
\setcounter{MaxMatrixCols}{30}

\newcommand{\nuc}[2]{$^{#2}$#1}
%++++++++++++++++++++++++++++++++++++++++

\begin{document}

%\section{An interesting distribution}
I am interested in the distribution of the random variable $x$ if I calculate it according to:

\begin{equation}
x = \sqrt{-log(r_1 r_2)}
\end{equation}
where $r_1$ and $r_2$ are random variables uniformly distributed between $0$ and $1$.

Consider the probability integral over all combinations of $r_1$ and $r_2$:

\begin{equation}
\label{simpleintegral}
\int_{0}^{1} \int_{0}^{1} dr_1 dr_2 = 1
\end{equation}

In order to understand the distribution of $x$ I will perform a change of variables in (\ref{simpleintegral}), setting:

\begin{equation}
\label{subs1}
\begin{aligned}
x&=\sqrt{-log(r_1 r_2)}\\
y&=r_2
\end{aligned}
\end{equation}
and then integrate out the $y$ dependence.

The inverse of the mapping in (\ref{subs1}) is:
\begin{equation}
\begin{aligned}
r_1&=\frac{1}{y}e^{-x^2}\\
r_2&=y
\end{aligned}
\end{equation}
Note that for a given $x$, we will need to have $y\ge e^{-x^2}$ so that $r_1$ will be in the interval $[0,1]$. The probability integral becomes:
\begin{equation}
\label{subs1integral}
\int_{0}^{1} \int_{0}^{1} dr_1 dr_2 = \int_{0}^{\infty}dx\int_{e^{-x^2}}^{1}|\mathbf{J}|dy
\end{equation}
where $\mathbf{J}$ is the Jacobian of the transform.

\begingroup
\renewcommand*{\arraystretch}{1.5}
\begin{equation}
\mathbf{J}=
\begin{vmatrix}
\frac{\partial r_1}{\partial x} & \frac{\partial r_1}{\partial y}\\
\frac{\partial r_2}{\partial x} & \frac{\partial r_2}{\partial y}
\end{vmatrix}
=
\begin{vmatrix}
-\frac{2 x}{y}e^{-x^2} & -\frac{1}{y^2}e^{-x^2}\\
0 & 1
\end{vmatrix}
=-\frac{2 x}{y}e^{-x^2}
\end{equation}
\endgroup

And so we get:
\begin{equation}
\begin{aligned}
\int_{0}^{\infty}dx\int_{e^{-x^2}}^{1}|\mathbf{J}|dy
&=\int_{0}^{\infty}2 x e^{-x^2} dx\int_{e^{-x^2}}^{1}\frac{1}{y}dy
=\int_{0}^{\infty}2 x e^{-x^2} dx (log(1)-log(e^{-x^2}))\\
&=\int_{0}^{\infty}2 x^3 e^{-x^2} dx=1
\end{aligned}
\end{equation}
I claim that we have integrated out a degree of freedom that we are not interested in and are left with the distribution of $x$.

\end{document}
